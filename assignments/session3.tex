\documentclass[12pt,a4paper]{article}
\usepackage{preamble}
\usepackage{graphicx}
\usepackage{fancyvrb}
\usepackage[bottom]{footmisc}
\newcommand\sessiontitle{Lab Session 3}
\newcommand\sessionsubtitle{Fourier transform}

%%%%%%%%%%%%%%%%%%%%%%%%%%%%%%%%%%%%%%%%%%
\begin{document}

%\section{Fourier transform}
%\label{task:fourier}

\noindent
Open the notebook \texttt{fourier.ipynb} and extend it as follows:

\begin{enumerate}
\item Explore the Fourier transform by following these steps:
\begin{enumerate}
    \item Load the first image\footnote{Source: \url{https://www.cellpose.org/dataset}} as given in the notebook, resize it to (256, 256) pixels using the \texttt{resize()} function, and show the result.
    \item Apply the \emph{fast Fourier transform} (FFT) to the image:
    \begin{Verbatim}[frame=single]
cell_ft = np.fft.fft2(cell_img)
    \end{Verbatim}
    Be careful to use the \texttt{fft2} function (not just \texttt{fft} without the ``\texttt{2}'') since we use a \emph{two}-dimensional image. The result should be a \emph{complex} array, so verify this by checking \texttt{cell\_ft.dtype} of the result of the FFT. -- \textbf{Hint:} The FFT is an algorithm that efficiently calculates the discrete Fourier transform (DFT).
    \item The result of the FFT is in the Cartesian form ($x = a + bi$, where ``$i$'' is the imaginary number). You can access the real part $a$ by \texttt{.real} and the imaginary part $b$ by \texttt{.imag}. Note that in Python the imaginary number $i$ is \texttt{j}. To interpret the image in the Fourier domain, take a look at the amplitude $r$ and phase $\phi$ considering the \emph{polar} form ($x = re^{i\phi}$). Extract the amplitude $r$ the phase $\phi$ using following code:
    \begin{Verbatim}[frame=single]
amplitude = np.abs(cell_ft)
phase = np.angle(cell_ft)
    \end{Verbatim}
    Display them side by side using the following code:
    \begin{Verbatim}[frame=single]
plt.figure()
plt.subplot(1, 2, 1)
plt.imshow(np.log(amplitude), 'gray')
plt.subplot(1, 2, 2)
plt.imshow(phase, 'gray')
    \end{Verbatim}
    \textbf{Hints:} The function \texttt{plt.subplot} adds a new subplot with the three parameters \textit{(i)} total number of rows, \textit{(ii)} total number of columns, and \textit{(iii)} at which position to place the image. Furthermore, show the \emph{logarithm} of the amplitude for better visibility of the huge range of values.
    \item Now, shift the zero components to the center of the spectral image using \texttt{np.fft.fftshift(cell\_ft)}. Extract and show amplitude and phase again.
    \item Restore the image using the \emph{inverse} fast Fourier transform (IFFT) by using the function \texttt{np.fft.ifft2} and show the restored image. The IFFT will also return a complex array. However, since the original image contains only \emph{real} numbers, the restored image can be accessed using the real values of the array. Please note, if you did a shift in the Fourier domain, you have to reverse this first with \texttt{np.fft.ifftshift} before applying the IFFT.
\end{enumerate}
\begin{samepage}
\item Study the role of the amplitude and the phase:
\begin{enumerate}
    \item Load the image \texttt{data/brain\_ct.png}\footnote{Source: \url{https://brain-development.org/ixi-dataset}}.
    \item Apply the FFT, do a shift, then extract and display phase and amplitude.
    \item Compose a merged image by ``restoring`` it from the amplitude of the brain image and the phase of the cell image. To this end, first convert the amplitude and phase to a Cartesian complex array using the pre-implemented function \texttt{to\_complex\_array}. Be careful to \emph{consistently} either use only the shifted or the non-shifted amplitude and phase, when merging. Reverse the shift if you used the shifted values and do the IFFT. Show the result.
    \item Repeat the previous step vice versa (use the amplitude of the cell image and the phase of the brain image).
    \item Conclude, \emph{what carries more information}, the phase or the amplitude?
\end{enumerate}
\end{samepage}
\item Apply a low-pass and a high-pass filter using the convolution theorem:
\begin{enumerate}
    \item Create a mask to use as low-pass filter on the shifted amplitudes. To this end, first create an array of the same \texttt{shape} as the image, filled with zeros. Insert a rectangle of filter window size with the value one to the mask at the center of the image. Display your mask to check if only the center contains ones. Your mask should look like the following image:
    \begin{figure}[h!]
            \centering
            \includegraphics[width=0.25\textwidth]{images/low_pass_filter.png}
        \end{figure}
    \item Also create a high-pass filter mask -- \textbf{Hint:} You can use ``one minus the low-pass mask'' to achieve this.
    \item Now apply your masks by multiplying them with the shifted amplitude of the brain image. Display the shifted amplitude without any filter, the low pass-filtered, and the high pass-filtered image side by side. Scale the amplitudes logarithmically.
    \item Merge the filtered amplitudes with the phase of the brain image to a complex array. Reverse the shift and apply the IFFT. Show the original image, the low pass-filtered, and the high pass-filtered merged images side by side.
    \item Repeat the above steps using different filter sizes and observe the differences.
\end{enumerate}
\begin{samepage}
\item Implement your own DFT function and compare the result as well as the run time to the implementation from \texttt{numpy}:
\begin{enumerate}
    \item Implement a re-usable function that computes the DFT, i.e. 
     \begin{gather}
        \texttt{dft}[u, v] = \frac{1}{MN} \sum_{x=0}^{M-1} \sum_{y=0}^{N-1} \texttt{img}[x, y] e^{-2j\pi (\frac{ux}{M}+\frac{vy}{N})}
        \label{eq:dft}
    \end{gather}
    \textbf{Hints:} First, create an array filled with complex values using
\begin{Verbatim}[frame=single]
dft = np.zeros(img.shape, dtype=complex)
    \end{Verbatim}
    Then, extract the dimensions of the image by \texttt{M,N = img.shape}. The formula~\eqref{eq:dft} calculates one entry of the \texttt{dft}-array, so you have to use \texttt{for}-loops to fill the whole array (which you initialized with zeros).
    \item Implement a re-usable function that computes the IDFT, i.e.
    \begin{gather}
        \texttt{idft}[u, v] = \sum_{x=0}^{M-1} \sum_{y=0}^{N-1} \texttt{dft}[x, y] e^{2j\pi (\frac{ux}{M}+\frac{vy}{N})}
    \end{gather}
    \item Try your implementation using the brain image and display amplitude and phase. -- \textbf{Hint:} You can still use the \texttt{fftshift} function after using your DFT.
    \item Reconstruct the image using your IDFT implementation and display the result.
    \item The FFT is an efficient algorithm to compute the DFT. Compare your results to those obtained using the FFT from \texttt{numpy}, which you used for the previous tasks. Apply the FFT as well as your DFT implementation to the brain image. Then, compare the results using \texttt{np.allclose}. -- \textbf{Note:} You must use the parameter \texttt{norm='forward'} for the \texttt{np.fft.fft2} function to obtain identical results. The reason is that there are slightly different definitions\footnote{\url{https://numpy.org/doc/stable/reference/routines.fft.html\#implementation-details}} of the DFT and using \texttt{norm='forward'} enforces the definition from Eq.~\eqref{eq:dft}.
    \item Compute the mean difference between the results of the two algorithms by \texttt{np.mean(np.abs(own\_dft - np\_fft))}. Afterwards, round the DFT and the FFT arrays using \texttt{np.round(array, 5)}, where 5 is the number of decimals to maintain.
    Compare the arrays using \texttt{(own\_dft == np\_fft).all()}. This statement returns \texttt{True} if all values of the arrays match. Select the number of decimals in such a way that you round as few as possible while the arrays are still matching.
    \item Compare the run times of your DFT implementation to the FFT implementation from \texttt{numpy} using the \texttt{\%timeit} command.
\end{enumerate}
\end{samepage}
\end{enumerate}

\end{document}
