\documentclass[12pt,a4paper]{article}
\usepackage{preamble}
\usepackage{lineno}
\let\AltKey=\Alt
\let\Alt=\relax
\usepackage{menukeys}
\newcommand\sessiontitle{Introduction}
\newcommand\sessionsubtitle{Preparations for the Second Week}

%%%%%%%%%%%%%%%%%%%%%%%%%%%%%%%%%%%%%%%%%%
\begin{document}
%\linenumbers

\begin{sloppypar}
\begin{description}
%\item[Course web-page:]~\\\url{http://www.bioquant.uni-heidelberg.de/research/groups/biomedical-}\\\url{computer-vision/teaching/python}
\item[Course web-page:]~\\\url{http://www.bioquant.uni-heidelberg.de/research/groups/biomedical-computer-vision/teaching/python}
\item[Overview of your GitHub Codespaces:] \url{https://github.com/codespaces}
\end{description}
\end{sloppypar}

If working in the BioQuant computer room, please \textbf{use Chrome instead of Firefox} as the web browser, because the installed version of Firefox is known to have issues with GitHub. Using Firefox is fine when working on your personal computer.

\section{Setting up your GitHub repository}
\label{task:preparation}
\begin{enumerate}
\item Open the course web-page in any web-browser of your choice.
\item Click on the link in ``Create a new GitHub repository by using this link''.
\item This will load another web-page entitled ``Create a new repository''. Leave everything on default and confirm the creation of the repository by clicking the ``Create repository'' button.
\item You should see a ``Generating your repository'' message for a few seconds and then be presented with an \textbf{overview of your repository}.
\end{enumerate}

\section{Firing up a GitHub Codespace}
\label{task:codespaces}
\begin{enumerate}
\item Open the \textbf{overview of your repository} on GitHub. This is the web-page that you landed on after completing Task~\ref{task:preparation}.
\item Click on the green ``Code`'' button, then select the ``Codespaces'' tab.
\item Click the green ``Create codespace on master'' button. This will load \textbf{VS Code} (Visual Studio Code) inside of your web-browser. Wait until everything is loaded, it may take about one minute.
\end{enumerate}

\vspace{-1em}
\begin{figure}[h!]
    \centering
    \includegraphics[width=0.7\textwidth]{images/codespaces.png}
\end{figure}

\textbf{Note:} If, at any time, \textbf{VS Code} behaves weirdly (e.g., complaining about missing extensions, not loading notebooks, or similar), try to simply reload the \textbf{VS Code} window. To do that, press \Ctrl+\keystroke{\shift}+\keystroke{P} (or \keystroke{\cmd}+\keystroke{\shift}+\keystroke{P} if you are on macOS) and type ``Reload window \Return''. Your work progress will be preserved.

%\newpage
\section{Working with VS Code and Jupyter notebooks}
\label{task:jupyter}

The left panel of \textbf{VS Code} shows an overview of \textbf{your local repository}. Right now it is identical to your GitHub repository. The assignments of this course are organized into several Jupyter notebooks. These are the "*.ipynb" files that you can see in your repository. By progressing from task to task, you will work with different notebooks (open a notebook by double clicking it).

In Jupyter notebooks, code cells can be run in an \emph{arbitrary order}. This is very helpful for experimenting and trying out new things. Still, an assignment in this course is \emph{only} considered ``finished'' when the results can be reproduced by re-running all code cells \emph{from top to bottom} by clicking the ``Run all'' button.

\section{Preserving your work progress with Git}
\label{task:git}

When finished, close your notebook. Changes are saved automatically within \textbf{your local repository}, but remember, that those changes will be lost when you close the codespace, unless you push them to your GitHub repository.

To do that, click on the ``Terminal'' tab at the bottom of \textbf{VS Code}, type the following Git command, and press \Return to execute it:
\begin{Verbatim}[frame=single]
git commit --all -m "Finish task 1.3"
\end{Verbatim}
The text ``"Finish task 1.3"'' is the \textbf{commit message}, which is arbitrary. Describe what changes you have done since your previous commit.

If you wanted, you could revert to any previous commit at a later time, and choosing an expressive message is convenient for finding the commit which you will be looking for. There also are some conventions for how a \textbf{commit message} should be formatted: It should tell in an ``imperative mood'' and as concisely as possible, what \emph{the committed changes} are supposed to do\footnote{From the official Git documentation: \texttt{https://git.kernel.org/pub/scm/git/git.git/tree\\/Documentation/SubmittingPatches?h=v2.36.1\#n181}}.

If done correctly, the output of the above command should be something like ``1 file changed, 12 insertions(+), 1 deletion(-)'', but the exact numbers may vary.

Finally, type ``"git push"'' and press \Return to \textbf{push the committed changes} to your GitHub repository. If done correctly, there should be multiple lines of output, concluding with a line similar to ``"1d2e403..a387645 master -> master"''.

\section{Create a cheat sheet for the second week}
Now you already know how to edit, commit, and push a Jupyter notebook.
\begin{enumerate}
    \item In preparation of the second week, create a cheat sheet of the most important takeaways from what you have learned in Datacamp. To do so, open the notebook \texttt{cheatsheet.ipynb} and complete it.
    \item Commit and push your changes, then close \textbf{VS Code} (the browser tab/window).
\end{enumerate}
\end{document}